\documentclass[a4paper]{article}

%opening

\title{\huge Assignment 2: Software Architecture
	\\COS 301 Team Alpha Project
	\\Version 1.0}

\author{\\\\Amy Lochner 14038600\\ Avinash Singh 14043778 \\
	Christiaan Nel 14029368\\ Christiaan Saaiman 12059138 \\
	Gerard van Wyk 14101263\\ Marc Antel 12026973\\
	Themba Mbhele 14007950
	\\
	\\
	\\\textit{https://github.com/AvinashSingh786/COS301-Alpha.git}
	\\
	\\
	\\ University Of Pretoria\\}

\date{March 2016}

\usepackage{graphicx}
\usepackage{float}
\usepackage{hyperref}
\begin{document}
	
	\maketitle

	\pagenumbering{gobble}
	\newpage

	\tableofcontents
	\pagenumbering{arabic}
	\newpage
	
	\section{Software Architecture Documentation}
	
	This document defines 
	
	\subsection{Architecture requirements}
In this section extract the architectural requirements from the software requirements including

	\subsubsection{Architectural scope}
	\subsubsection{Quality requirements}

	\subsubsection{Integration and access channel requirement}
		As it is possible that the system may have concurrent users, it is favourable that there be interfaces such as a web application or website, as well as a mobile platform applications for the various different mobile devices through which the system can be accessed.
		\\
	\subsubsection{Architectural constraints}
	It is desired that this system will encourage authors to collaborate with other authors on similar topics and to expand the users base knowledge of ongoing research projects.
	\\
	\\
	\\
	
	
	\section{Architectural patterns or styles}

	\subsection{Architectural tactics or strategies}

	\subsection{Use of reference architectures and frameworks}
	\subsection{Access and integration channels}
	Specify and quantify each of the quality requirements which are relevant to the system
	\begin{itemize}
		\item Performance - How well the system can cope with extreme load.
		\item Security - Minimising the possibility of leaking information, maintaining data integrity, and avoiding session hijacking.
		\item Maintainability - Can the system be managed without downtime.
		\item Scalability - Can the system be used for large amount of users without it affecting performance.
		\item Efficiency - Can the system be optimized to produce faster and better results.
		\item Flexibility - Can the system be easily changed or modified.
		\item Reliability - Is the system able to cope with the load in order to  provide constant access, ensure the system is always active and can provide all functionality.
		\item Integrability - Will the system be able to integrate with other technologies.
		\item User Friendly - Does a user easily understand how to use the system.
		\item Concurrency - Can multiple users perform actions at the same time.
		\item Low Cost, Reduced data usage - Is it suitable for users with low budget and capped Internet.
		\item Updatability - Can the system have version updates to introduce new features or functionality whilst maintaining old data.
		\\
			\begin{itemize}
				\item The different Web protocols used.
				\item API - UML Interfaces.
				\item Google Calender and Email integration.
				\item Mobile Scalability and functionality Integration.
				\item Venue and Publication integration.
				\\
			\end{itemize}
	\end{itemize}
	

	\subsection{Technologies}
	This specifies any constraints, the client may require, to be placed on the system architecture. Such constraints may be:
	\begin{itemize}
		\item HTML (Hypertext Markup Language)
		\item AJAX (Asynchronous JavaScript and XML)
		\item JavaEE (Java Platform Enterprise Edition)
		\item JavaScript (Functionality to HTML)
		\item PHP (Server Side Scripting)
		\item MySQL (Database Manager)
		\item Android (Mobile Devices)
		\item IOS (Mobile Devices)
		\item Apache (Web Server)
		\item Linux/Windows (Operating System)
		\\
		\\
	\end{itemize}
	\pagebreak
\end{document}
